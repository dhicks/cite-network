\documentclass{philpaper}

\title{Bibliometrics for Social Validation}
\author{Daniel J. Hicks}
\affiliation{AAAS S\&TP Fellow, Hosted in US EPA Office of Research and Development}
\email{hicks.daniel.j@gmail.com}

\section*{Abstract}

This paper introduces the concept of ``social validation'' to the bibliometric community, as well as a citation network method for assessing social validation, and a case study in the development of high-throughput toxicology methods at the US Environmental Protection Agency.  Social validation refers to the acceptance of novel research methods by a relevant scientific community; it is independent of the technical validation of methods, and is highly studied in history, philosophy, and social studies of science using qualitative methods.  The quantitative methods introduced here find that high-throughput toxicology methods are spread throughout a large and well-connected research community, which suggests high social validation.  Further assessment of social validation would involve qualitative methods, which are discussed in the conclusion.  


\section{Introduction}\label{sec.intro}

The validation of novel scientific methods takes place at two levels. The first level, or \emph{formal validation}, is the most familiar: methods are validated by showing that they are theoretically well-supported, their results can be replicated, and they agree with established methods (at least for cases where established and novel methods are expected to agree). The second level, or \emph{social validation}, concerns the acceptance of novel methods by the relevant scientific community. A novel method may have high formal validation when performed by its developers, but fail to be generally accepted by the scientific community if it is difficult to use, requires equipment or materials that are expensive or otherwise difficult to obtain, is very slow, depends on assumptions that are not widely accepted or mathematical techniques that are not widely understood, and so on.\footnote{``Social validation'' is my term for a concept that is widely studied in science and technology studies {[}STS{]} and philosophy of science. Key works on the role of social validation in scientific research include @Fleck1935, @KuhnSSR, @LonginoSSK. *[move this up to body]}

The EPA's Chemical Safety for Sustainability program is developing an array of novel methods in high-throughput toxicology {[}HTT{]} to address the lack of toxicity and exposure data for tens of thousands of commercial chemicals. For some particular uses, HTT methods have been shown to have high formal validation \autocite{Browne2015}, while formal validation is low in other use cases for currently-existing HTT assays and models \autocite{Silva2015, Janesick2016}. However, there has been no systematic study of the social validation of these methods. An analysis of the social validation of HTT could be useful whatever its findings: positive results (high social validation) could be useful when communicating the value of the research to audiences who are not familiar with HTT, while a careful analysis of negative results (low or mixed social validation) could help HTT researchers identify appropriate avenues for social validation.

*[recapitulate abstract + roadmap]

\section{Citation Network Analysis for Social Validation}\label{sec.citnet}

* conceptual model: low social validity corresponds to marginal subnet corresponds to high modularity

* network construction

* interpreting Q

* robustness by varying cutoffs

\section{Case Study: Social Validation of HTT}


\section{Further Assessment of Social Validation}

* quantitative: looking at CSS tasks or projects
* qualitative: content analysis of downstream citations
	- Catalini 2015


\end{document}